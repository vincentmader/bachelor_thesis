\documentclass[10 pt]{beamer}


%\usepackage[left=2cm, right=2cm, top=2cm, bottom=2cm]{geometry}
\usepackage{graphicx}
%\usepackage{hyperref}
%\usepackage{siunitx}
\usepackage{subfig}
%\usepackage[utf8]{inputenc}
%\usepackage{xcolor}

\setbeamertemplate{navigation symbols}{}
\graphicspath{{../figures/}}
%\sisetup{separate-uncertainty}

%% redefine paragraph to include newline after headline
%\newcommand{\myparagraph}[1]{\paragraph{#1}\mbox{} \}
%% redefine \vec command to have bold vectors
%\let\vec\mathbf
%% define text color commands
%\newcommand{\red}[1]{\textcolor{red}{#1}}
%\newcommand{\green}[1]{\textcolor{green}{#1}}
%\newcommand{\blue}[1]{\textcolor{blue}{#1}}


\begin{document}
  \begin{frame}

    \frametitle{Vincent Mader}

    \begin{itemize}
      \item doing Bachelor thesis under supervision of Bertram
      \item numerical simulations of planets on eccentric orbits with 
        FARGO2D1D algorithm \\
        $\Rightarrow$ how does eccentricity influence accretion 
        rate and gap structure?
    \end{itemize}

    \begin{figure}[h!]
      \centering
%      \begin{minipage}{.33\framewidth}
%        \centering
%        \subfloat[]{
%          \includegraphics[scale=.33]{frame_rotation/1.0mj_e.000/gas_dens_polar_out50.png}
%        }
%      \end{minipage}%
      \begin{minipage}{.5\framewidth}
        \centering
        \subfloat[radial gap structure for \\ various eccentricities]{
          \includegraphics[scale=.5]{frame_rotation/sigma_vs_r_and_e0_2500}
        }
      \end{minipage}%
      \begin{minipage}{.5\framewidth}
        \centering
        \subfloat[relative mass increase for various \\ \ \ \ eccentricities and planet masses]{
          \includegraphics[scale=.5]{frame_rotation/mpm0_vs_e0_and_m0}
        }
      \end{minipage}
    \end{figure}

  \end{frame}
\end{document}


