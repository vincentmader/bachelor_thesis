
\chapter{Numerical Techniques}

  \section{The FARGO2D1D Algorithm}
    The FARGO algorithm was originally introduced by 
    \red{who, 1999, what advantages relative to what?}. FARGO stands for 
    \textit{Fast Advection in Rotating Gaseous Objects}
    \cite{paper introducing fargo algorithm}. \\
    % fast Eulerian transport algorithm for differentially rotating disks
    \\
    FARGO2D1D is an extension of the standard version of FARGO, in which the 
    2D grid is surrounded by a simplified 1D grid, made of elementary rings 
    non azimuthally resolved.
    \cite{what is FARGO2D1D}
    
    \begin{figure}[h!]
      \centering
      \includegraphics[width=.4\textwidth]{fargo2D1D_grid.png}
      \caption{example of a grid used by the FARGO2D1D algorithm\cite{2D1D grid}}
      \label{}
    \end{figure} \ \\ 
    
    \red{
      parameters $G=1, M_\odot=1, R_0=1$ \\
      disk is modeled partially as a 2D array, partially as a 1D array \\
      $\Rightarrow$ make plot showing this (look in FARGO documentation?)
    }
    
    \begin{enumerate}
      \item n-body solver with 5th order Runge-Kutta algorithm
      \item fluid dyn for gas
    \end{enumerate}
    
    \subsection{Runge-Kutta}
      \begin{enumerate}
        \item family of implicit and explicit iterative methods
        \item developed around 1900 by German mathematicians Carl Runge and
          Wilhelm Kutta
        \item include the well-known routine called the Euler Method
        \item
        \item
      \end{enumerate}

