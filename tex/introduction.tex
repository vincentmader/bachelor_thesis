
% TODO: think about structure of introduction/theory

\paragraph{Protoplanetary disk}
  \begin{itemize}
    \item observations
    \item mathematical description
      \begin{itemize}
        \item fluid dynamics
        \item what kind of approximations are necessary?
      \end{itemize}
    \item numerical approach
  \end{itemize}

\myparagraph{FARGO2D1D algorithm}


\myparagraph{Accretion mechanisms}
  Two different mechanisms:
  \begin{enumerate}
    \item Machida et al. 2010, accretion \\
    runaway accretion when $m_{core}<m_{envelope}$
    \item Kley 1999, accretion
  \end{enumerate} 


\myparagraph{Hill sphere}
  The Hill sphere is the region around an astronomical body in which its
  gravitational influence dominates the attraction of sattelites
  \cite{def hill radius}. If a body with mass $m$ orbits a bigger object 
  with mass $M$ \red{with} a semi-major axis $a$ and eccentricity $e$, the 
  radius of the Hill sphere can be approximated
  (\red{why only approximately?}) by
  \begin{equation}
    r_H\approx a(1-e)\sqrt[3]{\frac{m}{3M}}
    \label{eq: def hill radius}
  \end{equation}

