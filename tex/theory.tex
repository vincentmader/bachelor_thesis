
\chapter{Theory}

  \section{Prerequisites}

    \subsection{Classical Theory of Gravity}
      
      \paragraph{Kepler's Laws}

        \begin{itemize}
          \item formulated by Johannes Kepler between 1609 and 1619 
          \item empirical laws from studies of Tycho Brahe's observational data
          \item first two laws were published in 1609, \textit{Astronomia Nova}
            \begin{enumerate}
              \item A planet travels around its parent star on an ellipse.
              \item The planet sweeps out equal areas in equals times 
            \end{enumerate}
          \item additionally, Kepler regognized that neither velocity nor 
            angular velocity are constant, but area velocity is 
            (closely related to angular momentum)
          \item ellipses are conical sections, characterized by the equation
            \begin{equation}
              \frac{x^2}{a^2}+\frac{y^2}{b^2}=1,\ \ \ \ \ \textnormal{with }a>b
            \end{equation}
          \item ellipse characterized by two numbers 
            ($a$ \& $b$ or one axis and $e$)
          \item Earth's orbit has an eccentricity of about $0.017$
            \begin{equation}
              e=\sqrt{1-\frac{b^2}{a^2}}
              \label{}
            \end{equation}
          \item \red{give equation for trajectory}
          \item circles can be seen as special cases of ellipses, or as a
            separate class of conical section
          \item can also be parabolas or hyperbolas ($e=1$ or $e\geq1$)
          \item third law published in 1619
          \item square of the orbital period of a planet is directly 
            proportional to the cube of the semi-major axis of its orbit.
        \end{itemize}

      \paragraph{Newton's Law of Gravitation}
        \begin{itemize}
          \item published in Isaac Newton's famous $Principia$
          \begin{equation}
            \vec{F}=G\cdot\frac{m_1\cdot m_2}{|\vec{r}_2-\vec{r}_1|^2}\cdot
            \frac{\vec{r}_2-\vec{r}_1}{|\vec{r}_2-\vec{r}_1|}
            \label{newtons_law_of_gravity}
          \end{equation}
          \item from this equation, Kepler's empirical laws could be 
            derived mathematically/theoretically (theoretical underpinning)
          \item inverse square law is responsible for first and third 
            Kepler law, conservation of angular momentum for second one
          \item Vis Viva equation \cite{vis_viva_equation}
        \end{itemize}

    \subsection{Fluid Dynamics}
  

  \section{Protoplanetary Disks}

    \begin{itemize}
      \item rotating circumstellar disk (newly formed star in the center)
      \item consists mostly of gas and dust
      \item most of the mass is in the star (\red{how much})
      \item often accompanied by jets
    \end{itemize}

    \subsection{Disk Formation}

      \begin{itemize}
        \item initial molecular gas cloud (nebular hypothesis)
        \item mainly \ch{H2}, about 98\%
        \item also small amounts of \ch{He} and \ch{Li} from Big Bang, as well 
          as miniscule amounts of heavier elements created in earlier
          generation(s) of stars
        \item nebular hypothesis formulated in the 1700s by Emanuel Swedenborg, 
          Immanuel Kant, and Pierre-Simon Laplace
        \item gravitational collapse (\red{which criteria/external influences})
        \item statistical motion averages out in favour of cloud's net 
          angular momentum 
        \item conservation of angular momentum: particles speed up as they 
          fall towards the center of the disk
        \item formation of relatively thin disk (gravity vs. centripetal force)
        \item can be seen partially as accretion disk onto star
        \item outcome: thin disk supported in vertical direction by gas pressure
          % Pringle, J.E. (1981). "Accretion discs in astrophysics". Annual 
          % Review of Astronomy and Astrophysics. 19: 137�162. 
          % Bibcode:1981ARA&A..19..137P. doi:10.1146/annurev.aa.19.090181.001033.
        \item can be modeled as ideal gas
      \end{itemize}

    \subsection{Star Formation in the Center of the Disk}

      \begin{itemize}
        \item accretion of gas onto star
        \item star forming in the center of the disk
        \item continues for about 3-10 million years
          % Mamajek, E.E.; Meyer, M.R.; Hinz, P.M.; Hoffmann, W.F.; Cohen, M. & Hora, J.L. (2004). "Constraining the Lifetime of Circumstellar Disks in the Terrestrial Planet Zone: A Mid-Infrared Survey of the 30 Myr old Tucana-Horologium Association". The Astrophysical Journal. 612 (1): 496�510. arXiv:astro-ph/0405271. Bibcode:2004ApJ...612..496M. doi:10.1086/422550
        \item 
        \item
        \item
      \end{itemize}

    \subsection{Geometry, Timescales \& other Properties}

      \begin{itemize}
        \item radii of up to 1000 AU
        \item much more wide than thick
        \item collapse takes about 100.000 years
          $\Rightarrow$ then, star has similar temperature to main sequence 
          star of same mass, becomes visible
        \item oldest disk ever observed: 25 million years
          % White, R.J. & Hillenbrand, L.A. (2005). "A Long-lived Accretion Disk around a Lithium-depleted Binary T Tauri Star". The Astrophysical Journal. 621 (1): L65�L68. arXiv:astro-ph/0501307. Bibcode:2005ApJ...621L..65W. doi:10.1086/428752. - Cain, Fraser; Hartmann, Lee (3 August 2005). "Planetary Disk That Refuses to Grow Up (Interview with Lee Hartmann about the discovery)". Universe Today. Retrieved 1 June 2013.
        \item after that, gas is either blown away by stellar wind or simply stops emitting radiation
        \item gas density can be modeled as 
          \begin{equation}
            \Sigma=\Sigma_0\bigg(\frac{r}{r_c}\bigg)^{-\gamma}
          \end{equation}
          with $\gamma\approx1$ (flaring index (?), geometry)
        \item cutoff radius (not $r_c$, is it?)
        \item gas is supported by gas pressure $\Rightarrow$ orbits slightly 
          slowly as it would if it were purely Keplerian
      \end{itemize}
  
  
  \section{Formation of Planets in the Disk}

  \begin{itemize}
    \item inner terrestrial planets, outer gaseous planets (gaseous planets 
      beyond the frost line much larger, since there is much more ice 
      than heavier elements $\Rightarrow$ could grow large enough to 
      capture abundant \ch{H2} and \ch{He})
    \item formation of circumplanetary disks (similar to protoplanetary disk, 
      much smaller scale, \red{how many AU approximately?})
    \item Roche lobe
    \item
    \item
  \end{itemize}

    \subsection{Planet Core Formation}

      \myparagraph{Formation of Planetesimals}

        \begin{itemize}
          \item collisions, gravitational capture
          \item still not entirely known how 
          \item electrostatic and gravitational interactions lead 
            accretion into planetesimals
          \item planetesimals are building blocks for both 
            terrestrial and giant planets
          \item problem: how does early core not fragment into smaller pieces 
            at collisions?  (meter size barrier, still unclear
        \end{itemize}

    \subsection{Accretion Mechanisms}

      \myparagraph{Machida Accretion}

      \myparagraph{Kley Accretion}



%\section{Definitions}
%
%  \myparagraph{Standard Gravitational Parameter}
%  The \textit{standard gravitational parameter} $\mu$ is defined as 
%  \begin{equation}
%    \mu:=G\cdot M
%    \label{}
%  \end{equation}
%  Here, $G$ is the Newtonian constant of gravitation and $M$ the mass of the
%  disk's central star. \red{Since computers like numbers close to 1, not using
%  SI, but ...} In these so called code units, both $G$ and $M$ will be set to 
%  $1$, which means that
%  \begin{equation}
%    \mu=1
%    \label{}
%  \end{equation}
%  in code units as well.
%
%
%\section{Laws of Planetary Motion}
%
%  
%  \myparagraph{The \textit{vis viva} equation}
%  To model the trajectories of orbiting bodies, the following relation can be
%  used\cite{vis_viva_equation}:
%  \begin{equation}
%    v^2=GM\bigg(\frac{2}{r}-\frac{1}{a}\bigg)
%    \label{}
%  \end{equation}
%  For a circular orbit, the approximation $r\approx a$ holds. With
%  $\Omega=\frac{v}{r}$ this leads to
%  \begin{equation}
%    \Omega^2=\frac{GM_\odot}{a^3}
%  \end{equation}
%  which can also be easily derived by equating the gravitational and 
%  centripetal forces.
%  If, on the other hand, the eccentricity $e\neq0$, then
%  \begin{equation}
%    \Omega^2=\frac{GM}{r^2}\bigg(\frac{2}{r}-\frac{1}{a}\bigg)
%  \end{equation}
%
%
%\section{Fluid Dynamics}
%
%
%\section{Protoplanetary Disks}
%
%  \subsection{General Model/Approximations \red{(which headline?)}}
%    \begin{itemize}
%      \item two-dimensional, non-self-gravitating protostellar accretion disk
%      \item embedded: $\sim$ Jupiter-sized protoplanet on an orbit \red{(ecc.?)}
%      \item goal: solve hydrodynamical equations (Navier-Stokes)
%      \item FARGO2D1D: n-body solver for planet motion
%      \item $+$ accretion \red{how exactly? taken from inside Roche lobe? (for Kley yes)} \\
%        \red{what function determines the accretion?}
%      \item \red{different values of $H/r$, $\alpha_{visc.}$}
%    \end{itemize}
%    \red{...}
%    \begin{itemize}
%      \item observations
%      \item mathematical description
%        \begin{itemize}
%          \item fluid dynamics
%          \item what kind of approximations are necessary?
%        \end{itemize}
%      \item numerical approach
%    \end{itemize}
%
%    \myparagraph{\red{Gap Opening}}
%    \begin{equation}
%      c_s=H\cdot\Omega
%    \end{equation}
%    \begin{equation}
%      P=c_s^2\Sigma  
%    \end{equation}
%    
%
%
%  \newpage
%  \subsection{Planets In The Disk}
%
%    \myparagraph{The Hill Sphere}
%      The Hill sphere is the region around an astronomical body in which its
%      gravitational influence dominates the attraction of sattelites
%      \cite{def hill radius}. If a body with mass $m$ orbits a larger mass
%      $M$ with a semi-major axis $a$ and eccentricity $e$, the 
%      radius of the Hill sphere can be approximated by
%      \begin{equation}
%        r_H\approx a(1-e)\sqrt[3]{\frac{m}{3M}}
%        \textnormal{\ \ \red{(why only approximately?)}}
%        \label{eq: def hill radius}
%      \end{equation}
%
%    \myparagraph{The Roche Limit \red{(relevant? Roche Lobe vs. Roche Limit?)}}
%      Let us consider two identical masses $m$ in orbit around a central mass
%      $M$. The central body exerts a gravitational force on both of the
%      orbiting masses, which can be expressed as
%      \begin{equation}
%        F_i=G\cdot\frac{M\cdot m}{r_i^2}
%      \end{equation}
%      Here, \red{$G$ labels the Newtonian gravitational constant and} $r_i$ is 
%      the distance from the central body to one of the two orbiting masses,
%      characterized by the index $i\in\{1,2\}$. The \textit{tidal force}
%      \begin{equation}
%        \Delta F=F_2-F_1
%      \end{equation}
%      is defined as the difference of these two forces. Whether or not the two
%      bodies form a bound system relies on whether or not the tidal force is 
%      greater than the mutual gravitational attraction between the two bodies, 
%      which can be written as 
%      \begin{equation}
%        F_{12}=G\cdot \frac{m^2}{r_{21}^2},
%      \end{equation}
%      where $r_{12}$ denotes the distance between the two orbiting masses. The 
%      Roche limit is reached when the following relation holds:
%      \begin{equation}
%        F_{12}=\Delta F,
%      \end{equation}
%      which can be expanded into
%      \begin{equation}
%        G\cdot M\cdot m\cdot(\frac{1}{r_2^2}-\frac{1}{r_1^2})
%        =G\cdot \frac{m^2}{r_{12}^2}
%      \end{equation}
%      and then simplified to 
%      \begin{equation}
%        M\cdot(\frac{1}{r_2^2}-\frac{1}{r_1^2})
%        =m\cdot\frac{1}{r_{12}^2}
%      \end{equation}
%      \begin{equation}
%        r_{12}^2\frac{r_1^2-r_2^2}{r_1^2\cdot r_2^2}
%        =\frac{m}{M}
%      \end{equation}
%      \red{...} \\
%      \red{masses will not be bound if $r_{12}\leq r_{crit.}$} \\
%      \red{(later used when FARGO2D1D accretes material from inside Roche lobe?)}
%
%
%    \newpage
%    \subsection{Mass Accretion Mechanisms}
%    
%      Two different mechanisms:
%      \begin{enumerate}
%        \item Machida et al. 2010, accretion \\
%        runaway accretion when $m_{core}<m_{envelope}$
%        \item Kley 1999, accretion
%      \end{enumerate} 
%
%      \myparagraph{\red{Kley}}
%      \begin{itemize}
%        \item suggested by Willy Kley in 1999 paper
%        \item way to implement numerical accretion algorithm 
%        \item in each time step, mass is removed from the disk 
%          (more precisely, the Roche lobe) and added to the planet
%      \end{itemize}
%      \begin{equation}
%        m_{disk}(t+dt)=m_{disk}(t)-dm
%      \end{equation}
%      \begin{equation}
%        m_{planet}(t+dt)=m_{planet}(t)+dm
%      \end{equation}
%      \begin{equation}
%        dm=f_{red}\cdot S_{acc}\cdot\Sigma(r,t)\cdot f_{acc}\cdot dt
%      \end{equation}
%
%
%      \myparagraph{\red{Machida}}
%      
%      
%
