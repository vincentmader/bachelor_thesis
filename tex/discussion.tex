
\chapter{Discussion}
  The number of discovered exoplanets has grown drastically in recent 
  years. This opens up the opportunity to constrain models of the planetary 
  formation processes using statistical properties extracted from the 
  population of discovered planets. A promising way to link those models with 
  observations is with so-called planet population synthesis, where a population 
  of planets is initialized with randomly distributed parameters. The population 
  is then allowed to evolve over time and can subsequently be compared with 
  observation data to assess the accuracy of the involved models. As such, 
  the most important part of any study using population synthesis is arguably 
  the planet formation model, which takes in a distribution of initial 
  parameters and outputs the properties of the emerging planetary system. \\
  \\
  Studies utilizing planet population synthesis this have been conducted by 
  e.g. \citeauthor{Ida_2018} (\citeyear{Ida_2018}), \citeauthor{Benz_2014} 
  (\citeyear{Benz_2014}) and \citeauthor{Mordasini_2012} 
  (\citeyear{Mordasini_2012}). When formulating a routine which describes the 
  accretion rate of disk material onto a planet, none of these mentioned 
  studies take the planet's orbital eccentricity into account. \\
  \\
  % One of the randomly initialized parameters is orbital eccentricity. Most 
  % \red{...} of population synthesis do not take the eccentricity into 
  % account when determining the rate of gas accretion onto a planet. \\
  % \red{cite{cite, Benz}} \\
  As could be shown in this thesis, the orbital eccentricity can play a 
  large role in the accretion process. Thus, when trying to construct an 
  accurate planet formation model, the accretion routine should be modified in 
  such a way as to take these effects into account. It could also be 
  beneficial to do this when running N-body simulations like the ones 
  that have been reported by \citeauthor{Bitsch_2019} (\citeyear{Bitsch_2019}).
  They also do not take the influence of the orbital eccentricity into 
  account for the formulation of their mass accretion routine. Since 
  planet-planet interaction is a viable mechanism for the creation of highly 
  elliptical orbits, N-body simulations likely lead to a family of bodies 
  orbiting their parent star on orbits with $e\neq0$. Therefore, these and 
  similar studies could benefit from incorporating 
  an eccentricity dependency into their accretion routine. \\
  \\
  The disk model that was used in this thesis is a highly simplified one. 
  While we could show the influence of orbital eccentricities on the rate 
  of accretion in a qualitative way, it would be advantageous to make use of a 
  more sophisticated model, when one really wants to determine an accurate 
  relationship between $e$ and $\dot{m}$.
  % it would be advantageous to make use of a 
  % more sophisticated approach. 
  Obvious ways of improvement include the 
  utilization of larger values for the grid resolution and integration time.
  This thesis also treated the disk as being both locally isothermal and 
  non-flared, as well as ignoring the influence of physical processes like 
  stellar radiation and magneto-hydro-dynamics in the disk, among others. 
  A simulation of protoplanetary disks using a more sophisticated approach 
  could be done to display the relationship between eccentricity and 
  accretion in a more quantitatively accurate manner, which after the fitting 
  of a suitable function would then be able to be used in future studies, 
  improving their results.

  % To be able to improve the accuracy of N-body simulations or population 
  % synthesis with the results of this thesis, it would be 


  % du kannst auch sagen, dass das wichtig wird für n-body simulationen, siehe 
  % where is eccentricity high? young systems, gas phase, n body, older systems?

  % \begin{itemize}
  %   \item used very simplified model of the disk 
  %   \item lots of stuff neglected (MHD, radiation, temperature differences...)
  %   \item migration: type I, radiation...
  %   \item still, could show the effect of orbital eccentricity on accretion
  %   \item 
  %   \item most models of the early processes inside protoplanetary disks 
  %     assume that the eccentricity has a negligible effect on accretion 
  %     \red{\cite{cite}}
  %   \item planet population synthesis (Bitsch 2018) \\
  %     Ndugu et al. 2018, Mordasini et al. 2012, Ida et al. 2018
  %     \url{http://www.mpia.de/homes/ppvi/chapter/benz.pdf}
  %     \citeauthor{Benz_2014} (\citeyear{Benz_2014})
  %   \item eccentricity steigt auch durch planet-planet streuuung
  %   \item 
  %   \item only relevant for planets massive enough to form a gap, yet still 
  %     in a gap with enough free material to allow accretion
  %   \item accurate models of proto-planetary disks should incorporate this 
  %     into their accretion routines, especially when dealing with young 
  %     circumstellar disks, where eccentricities are still much higher than 
  %     in the Solar System
  %     \red{(talk about how eccentricity is high also in older systems)}
  %   \item
  % \end{itemize}

